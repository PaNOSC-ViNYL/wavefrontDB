   

In this section, we will demonstrate how to propagate a simple Gaussian
wavefront between two planes through vacuum. This is the simplest case
of coherent wavefront propagation.

    \begin{tcolorbox}[breakable, size=fbox, boxrule=1pt, pad at break*=1mm,colback=cellbackground, colframe=cellborder]
\prompt{In}{incolor}{7}{\boxspacing}
\begin{Verbatim}[commandchars=\\\{\}]
\PY{c+c1}{\PYZsh{} Import all SimEx modules}
\PY{k+kn}{from} \PY{n+nn}{SimEx}\PY{n+nn}{.}\PY{n+nn}{Calculators}\PY{n+nn}{.}\PY{n+nn}{WavePropagator} \PY{k+kn}{import} \PY{n}{WavePropagator}
\PY{k+kn}{from} \PY{n+nn}{SimEx}\PY{n+nn}{.}\PY{n+nn}{Calculators}\PY{n+nn}{.}\PY{n+nn}{GaussianPhotonSource} \PY{k+kn}{import} \PY{n}{GaussianPhotonSource}
\PY{k+kn}{from} \PY{n+nn}{SimEx}\PY{n+nn}{.}\PY{n+nn}{Parameters}\PY{n+nn}{.}\PY{n+nn}{WavePropagatorParameters} \PY{k+kn}{import} \PY{n}{WavePropagatorParameters}
\PY{k+kn}{from} \PY{n+nn}{SimEx}\PY{n+nn}{.}\PY{n+nn}{Parameters}\PY{n+nn}{.}\PY{n+nn}{GaussWavefrontParameters} \PY{k+kn}{import} \PY{n}{GaussWavefrontParameters}

\PY{k+kn}{from} \PY{n+nn}{SimEx}\PY{n+nn}{.}\PY{n+nn}{Utilities}\PY{n+nn}{.}\PY{n+nn}{Units} \PY{k+kn}{import} \PY{n}{electronvolt}\PY{p}{,} \PY{n}{meter}\PY{p}{,} \PY{n}{joule}\PY{p}{,} \PY{n}{radian}
\PY{k+kn}{import} \PY{n+nn}{numpy}
\end{Verbatim}
\end{tcolorbox}

    \hypertarget{setup-the-initial-wavefront}{%
\subsubsection{Setup the initial
wavefront}\label{setup-the-initial-wavefront}}

We first create a wavefront with a Gaussian intensity distribution. To
this end, we use the \texttt{GaussianPhotonSource} Calculator and it's
corresponding parameter, the \texttt{GaussWavefrontParameters}.

    We first describe the wavefront parameters which will then be used to
create the wavefront itself:

    \begin{tcolorbox}[breakable, size=fbox, boxrule=1pt, pad at break*=1mm,colback=cellbackground, colframe=cellborder]
\prompt{In}{incolor}{8}{\boxspacing}
\begin{Verbatim}[commandchars=\\\{\}]
\PY{n}{wavefront\PYZus{}parameters} \PY{o}{=} \PY{n}{GaussWavefrontParameters}\PY{p}{(}\PY{n}{photon\PYZus{}energy}\PY{o}{=}\PY{l+m+mf}{8.0e3}\PY{o}{*}\PY{n}{electronvolt}\PY{p}{,}
                                                \PY{n}{photon\PYZus{}energy\PYZus{}relative\PYZus{}bandwidth}\PY{o}{=}\PY{l+m+mf}{1e\PYZhy{}3}\PY{p}{,}
                                                \PY{n}{beam\PYZus{}diameter\PYZus{}fwhm}\PY{o}{=}\PY{l+m+mf}{1.0e\PYZhy{}4}\PY{o}{*}\PY{n}{meter}\PY{p}{,}
                                                \PY{n}{pulse\PYZus{}energy}\PY{o}{=}\PY{l+m+mf}{2.0e\PYZhy{}6}\PY{o}{*}\PY{n}{joule}\PY{p}{,}
                                                \PY{n}{number\PYZus{}of\PYZus{}transverse\PYZus{}grid\PYZus{}points}\PY{o}{=}\PY{l+m+mi}{400}\PY{p}{,}
                                                \PY{n}{number\PYZus{}of\PYZus{}time\PYZus{}slices}\PY{o}{=}\PY{l+m+mi}{12}\PY{p}{,}
                                                \PY{n}{z}\PY{o}{=}\PY{l+m+mi}{100}\PY{o}{*}\PY{n}{meter}\PY{p}{,}
                                                \PY{p}{)}
\end{Verbatim}
\end{tcolorbox}

    The parameters of the wavefront constructor are self explaining. Note
that \(z\), the longitudinal wavefront coordinate is set to \(100\) m,
i.e.~far enough away from the source position to justify a far-field
treatment.

    Now, we use these\texttt{wavefront\_parameters} to initialize the Photon
Source.

    \begin{tcolorbox}[breakable, size=fbox, boxrule=1pt, pad at break*=1mm,colback=cellbackground, colframe=cellborder]
\prompt{In}{incolor}{25}{\boxspacing}
\begin{Verbatim}[commandchars=\\\{\}]
\PY{n}{photon\PYZus{}source} \PY{o}{=} \PY{n}{GaussianPhotonSource}\PY{p}{(}\PY{n}{wavefront\PYZus{}parameters}\PY{p}{,} \PY{n}{input\PYZus{}path}\PY{o}{=}\PY{l+s+s2}{\PYZdq{}}\PY{l+s+s2}{/dev/null}\PY{l+s+s2}{\PYZdq{}}\PY{p}{,} \PY{n}{output\PYZus{}path}\PY{o}{=}\PY{l+s+s2}{\PYZdq{}}\PY{l+s+s2}{initial\PYZus{}wavefront.h5}\PY{l+s+s2}{\PYZdq{}}\PY{p}{)}
\end{Verbatim}
\end{tcolorbox}

    Let's calculate the initial wavefront and visualize it. Running the
\texttt{backengine} method of \texttt{photon\_source}, initializes the
data fields and returns the Rayleigh length.

    \begin{tcolorbox}[breakable, size=fbox, boxrule=1pt, pad at break*=1mm,colback=cellbackground, colframe=cellborder]
\prompt{In}{incolor}{26}{\boxspacing}
\begin{Verbatim}[commandchars=\\\{\}]
\PY{n}{photon\PYZus{}source}\PY{o}{.}\PY{n}{backengine}\PY{p}{(}\PY{p}{)}
\end{Verbatim}
\end{tcolorbox}

    \begin{Verbatim}[commandchars=\\\{\}]
146.22875114783653
    \end{Verbatim}

    We can retrieve the wavefront data from the calculator:

    \begin{tcolorbox}[breakable, size=fbox, boxrule=1pt, pad at break*=1mm,colback=cellbackground, colframe=cellborder]
\prompt{In}{incolor}{27}{\boxspacing}
\begin{Verbatim}[commandchars=\\\{\}]
\PY{n}{wavefront} \PY{o}{=} \PY{n}{photon\PYZus{}source}\PY{o}{.}\PY{n}{data}
\end{Verbatim}
\end{tcolorbox}

    Let's visualize the wavefront using the WPG utilities.

    \begin{tcolorbox}[breakable, size=fbox, boxrule=1pt, pad at break*=1mm,colback=cellbackground, colframe=cellborder]
\prompt{In}{incolor}{28}{\boxspacing}
\begin{Verbatim}[commandchars=\\\{\}]
\PY{k+kn}{from} \PY{n+nn}{wpg} \PY{k+kn}{import} \PY{n}{wpg\PYZus{}uti\PYZus{}wf} \PY{k}{as} \PY{n}{wpg\PYZus{}utils}
\end{Verbatim}
\end{tcolorbox}

    First plot the intensity distribution

    \begin{tcolorbox}[breakable, size=fbox, boxrule=1pt, pad at break*=1mm,colback=cellbackground, colframe=cellborder]
\prompt{In}{incolor}{56}{\boxspacing}
\begin{Verbatim}[commandchars=\\\{\}]
\PY{n}{wpg\PYZus{}utils}\PY{o}{.}\PY{n}{plot\PYZus{}intensity\PYZus{}map}\PY{p}{(}\PY{n}{wavefront}\PY{p}{)}
\end{Verbatim}
\end{tcolorbox}

    \begin{Verbatim}[commandchars=\\\{\}]
R-space
(400,) (400,)
FWHM in x = 1.147e-04 m.
FWHM in y = 1.147e-04 m.
    \end{Verbatim}

    \begin{center}
    \adjustimage{max size={0.9\linewidth}{0.9\paperheight}}{propagation_example_01_files/propagation_example_01_15_1.png}
    \end{center}
    { \hspace*{\fill} \\}
    
    \hypertarget{plot-intensity-distribution-in-q-space}{%
\subsection{Plot intensity distribution in
q-space}\label{plot-intensity-distribution-in-q-space}}

Then, plot the intensity profile in reciprocal (\(q\)-)space.

    \begin{tcolorbox}[breakable, size=fbox, boxrule=1pt, pad at break*=1mm,colback=cellbackground, colframe=cellborder]
\prompt{In}{incolor}{12}{\boxspacing}
\begin{Verbatim}[commandchars=\\\{\}]
\PY{n}{wpg\PYZus{}utils}\PY{o}{.}\PY{n}{plot\PYZus{}intensity\PYZus{}qmap}\PY{p}{(}\PY{n}{wavefront}\PY{p}{)}
\end{Verbatim}
\end{tcolorbox}

    \begin{Verbatim}[commandchars=\\\{\}]
Q-space
\{'fwhm\_x': 6.657123479604694e-07, 'fwhm\_y': 6.657123479604694e-07\}
Q-space
(400,) (400,)
    \end{Verbatim}

    \begin{center}
    \adjustimage{max size={0.9\linewidth}{0.9\paperheight}}{propagation_example_01_files/propagation_example_01_17_1.png}
    \end{center}
    { \hspace*{\fill} \\}
    
    Next, plot the power as a function of time integrated over the
transverse dimensions.

    \begin{tcolorbox}[breakable, size=fbox, boxrule=1pt, pad at break*=1mm,colback=cellbackground, colframe=cellborder]
\prompt{In}{incolor}{13}{\boxspacing}
\begin{Verbatim}[commandchars=\\\{\}]
\PY{n}{wpg\PYZus{}utils}\PY{o}{.}\PY{n}{integral\PYZus{}intensity}\PY{p}{(}\PY{n}{wavefront}\PY{p}{)}
\end{Verbatim}
\end{tcolorbox}

    \begin{center}
    \adjustimage{max size={0.9\linewidth}{0.9\paperheight}}{propagation_example_01_files/propagation_example_01_19_0.png}
    \end{center}
    { \hspace*{\fill} \\}
    
    \begin{center}
    \adjustimage{max size={0.9\linewidth}{0.9\paperheight}}{propagation_example_01_files/propagation_example_01_19_1.png}
    \end{center}
    { \hspace*{\fill} \\}
    
    \begin{Verbatim}[commandchars=\\\{\}]
number of meaningful slices: 7
Pulse energy 2e-06 J
    \end{Verbatim}

            \begin{tcolorbox}[breakable, size=fbox, boxrule=.5pt, pad at break*=1mm, opacityfill=0]
\prompt{Out}{outcolor}{13}{\boxspacing}
\begin{Verbatim}[commandchars=\\\{\}]
1059337216.0
\end{Verbatim}
\end{tcolorbox}
        
    Finally, we check the sampling quality.

    \begin{tcolorbox}[breakable, size=fbox, boxrule=1pt, pad at break*=1mm,colback=cellbackground, colframe=cellborder]
\prompt{In}{incolor}{15}{\boxspacing}
\begin{Verbatim}[commandchars=\\\{\}]
\PY{n+nb}{print}\PY{p}{(}\PY{n}{wpg\PYZus{}utils}\PY{o}{.}\PY{n}{check\PYZus{}sampling}\PY{p}{(}\PY{n}{wavefront}\PY{p}{)}\PY{p}{)}
\end{Verbatim}
\end{tcolorbox}

    \begin{Verbatim}[commandchars=\\\{\}]
WAVEFRONT SAMPLING REPORT
+----------+---------+---------+---------+---------+---------+---------+--------
-+
|x/y       |FWHM     |px       |ROI      |R        |Fzone    |px*7     |px*10
|
+----------+---------+---------+---------+---------+---------+---------+--------
-+
|Horizontal|1.147e-04|6.386e-06|2.548e-03|1.772e+02|1.196e-04|4.470e-05|6.386e-0
5|
|Vertical
|1.147e-04|6.386e-06|2.548e-03|1.772e+02|1.196e-04|4.470e-05|6.386e-05|
+----------+---------+---------+---------+---------+---------+---------+--------
-+

Horizontal Fresnel zone extension NOT within [7,10]*pixel\_width -> Check pixel
width."
Vertical Fresnel zone extension NOT within [7,10]*pixel\_height -> Check pixel
width."
Horizontal ROI > 3* FWHM(x) -> OK
Horizontal ROI > 3* FWHM(y) -> OK
Focus sampling: FWHM > 10*px

END OF REPORT
    \end{Verbatim}

    The wavefront quality report is somewhat suboptimal because the Fresnel
zone extension is below the recommended minimimum of 7 times the
sampling (pixel) size. We will proceed anyway and check after
propagation if this is a problem.

Our next step is to propagate the wavefront through 100 m of empty
space. Let's estimate the beam size using the beam waist formula.

    The beam waist size (rms width of the E-field distribution, or distance
from beam axis where intensity drops to \(1/e^2\) of its on-axis value)
is given by the expression

\(w(z) = w\_0 \sqrt{1 + \left(\frac{z}{R}\right)^2}\), with the
Rayleigh length \(R = \frac{\pi w_0^2}{\lambda}\), \(\lambda\) is the
central wavelength of the laser (ca,. 0.15 nm at 8 keV.)

Our Gaussian source distribution has a FWHM of \(10^{-4} m\). The FWHM
and the beam waist radius as defined above are related to each other
through \(\mathrm{FWHM}(z) = \sqrt{2 \ln 2} w(z) \simeq 1.177 w(z)\)

Beam waist at z=0:
\(w_0 = 100\,\mathrm{\mu m} / 1.177 \simeq 84.9\,\mathrm{\mu m}\).\\
The Rayleigh length for our Gaussian beam is
\(R = \frac{\pi w_0^2}{0.15 \mathrm{nm}} \simeq 151\,\mathrm{m}\).

At \(z=100\,\mathrm{m}\), we therefore expect a FWHM of
\(100\,\mu\mathrm{m}\cdot\sqrt{1.+\left(100/151\right)^2} \simeq 119.9 \mu\mathrm{m}\).

    \hypertarget{setup-the-beamline}{%
\subsection{Setup the beamline}\label{setup-the-beamline}}

We are now ready to specify the beamline. Import the corresponding
classes,

    \begin{tcolorbox}[breakable, size=fbox, boxrule=1pt, pad at break*=1mm,colback=cellbackground, colframe=cellborder]
\prompt{In}{incolor}{30}{\boxspacing}
\begin{Verbatim}[commandchars=\\\{\}]
\PY{k+kn}{from} \PY{n+nn}{wpg} \PY{k+kn}{import} \PY{n}{Beamline}\PY{p}{,} \PY{n}{optical\PYZus{}elements}\PY{p}{,} \PY{n}{srwlib}
\end{Verbatim}
\end{tcolorbox}

    And initialize

    \begin{tcolorbox}[breakable, size=fbox, boxrule=1pt, pad at break*=1mm,colback=cellbackground, colframe=cellborder]
\prompt{In}{incolor}{33}{\boxspacing}
\begin{Verbatim}[commandchars=\\\{\}]
\PY{n}{beamline} \PY{o}{=} \PY{n}{Beamline}\PY{p}{(}\PY{p}{)}
\PY{n}{free\PYZus{}space} \PY{o}{=} \PY{n}{optical\PYZus{}elements}\PY{o}{.}\PY{n}{Drift}\PY{p}{(}\PY{n}{\PYZus{}L}\PY{o}{=}\PY{l+m+mi}{99}\PY{p}{,} \PY{n}{\PYZus{}treat}\PY{o}{=}\PY{l+m+mi}{1}\PY{p}{)}
\PY{n}{free\PYZus{}space\PYZus{}propagation\PYZus{}parameters} \PY{o}{=} \PY{n}{optical\PYZus{}elements}\PY{o}{.}\PY{n}{Use\PYZus{}PP}\PY{p}{(}\PY{n}{semi\PYZus{}analytical\PYZus{}treatment}\PY{o}{=}\PY{l+m+mi}{1}\PY{p}{)}
\end{Verbatim}
\end{tcolorbox}

    Finaly, assemble the beamline

    \begin{tcolorbox}[breakable, size=fbox, boxrule=1pt, pad at break*=1mm,colback=cellbackground, colframe=cellborder]
\prompt{In}{incolor}{34}{\boxspacing}
\begin{Verbatim}[commandchars=\\\{\}]
\PY{n}{beamline}\PY{o}{.}\PY{n}{append}\PY{p}{(}\PY{n}{free\PYZus{}space}\PY{p}{,} \PY{n}{free\PYZus{}space\PYZus{}propagation\PYZus{}parameters}\PY{p}{)}
\end{Verbatim}
\end{tcolorbox}

    To start the propagation, we transform the wavefront to frequency space,
run the beamline's \texttt{.propagate()} method and backtransform to
temporal representation.

    \begin{tcolorbox}[breakable, size=fbox, boxrule=1pt, pad at break*=1mm,colback=cellbackground, colframe=cellborder]
\prompt{In}{incolor}{26}{\boxspacing}
\begin{Verbatim}[commandchars=\\\{\}]
\PY{n}{srwlib}\PY{o}{.}\PY{n}{srwl}\PY{o}{.}\PY{n}{SetRepresElecField}\PY{p}{(}\PY{n}{wavefront}\PY{o}{.}\PY{n}{\PYZus{}srwl\PYZus{}wf}\PY{p}{,} \PY{l+s+s1}{\PYZsq{}}\PY{l+s+s1}{f}\PY{l+s+s1}{\PYZsq{}}\PY{p}{)}
\PY{n}{beamline}\PY{o}{.}\PY{n}{propagate}\PY{p}{(}\PY{n}{wavefront}\PY{p}{)}
\PY{n}{srwlib}\PY{o}{.}\PY{n}{srwl}\PY{o}{.}\PY{n}{SetRepresElecField}\PY{p}{(}\PY{n}{wavefront}\PY{o}{.}\PY{n}{\PYZus{}srwl\PYZus{}wf}\PY{p}{,} \PY{l+s+s1}{\PYZsq{}}\PY{l+s+s1}{t}\PY{l+s+s1}{\PYZsq{}}\PY{p}{)}
\end{Verbatim}
\end{tcolorbox}

            \begin{tcolorbox}[breakable, size=fbox, boxrule=.5pt, pad at break*=1mm, opacityfill=0]
\prompt{Out}{outcolor}{26}{\boxspacing}
\begin{Verbatim}[commandchars=\\\{\}]
<wpg.srw.srwlib.SRWLWfr at 0x2af03f55c4d0>
\end{Verbatim}
\end{tcolorbox}
        
    The wavefront has now been propagated through the beamline. As above, we
can visualize the intensity profiles in various projections and Fourier
components:

    \hypertarget{real-space-transverse-intensity-profile}{%
\subsubsection{Real space transverse intensity
profile:}\label{real-space-transverse-intensity-profile}}

    \begin{tcolorbox}[breakable, size=fbox, boxrule=1pt, pad at break*=1mm,colback=cellbackground, colframe=cellborder]
\prompt{In}{incolor}{21}{\boxspacing}
\begin{Verbatim}[commandchars=\\\{\}]
\PY{n}{wpg\PYZus{}utils}\PY{o}{.}\PY{n}{plot\PYZus{}intensity\PYZus{}map}\PY{p}{(}\PY{n}{wavefront}\PY{p}{)}
\end{Verbatim}
\end{tcolorbox}

    \begin{Verbatim}[commandchars=\\\{\}]
R-space
(400,) (400,)
FWHM in x = 1.282e-04 m.
FWHM in y = 1.282e-04 m.
    \end{Verbatim}

    \begin{center}
    \adjustimage{max size={0.9\linewidth}{0.9\paperheight}}{propagation_example_01_files/propagation_example_01_34_1.png}
    \end{center}
    { \hspace*{\fill} \\}
    
    The propagated wavefront has a
\(\mathrm{FWHM} = 128.2\,\mathrm{\mu m}\). Considering that the FWHM is
measured by counting pixels in the intensity profile, this is in rather
good agreement with the analytical calculation above which yielded
\(119\,\mathrm{\mu m}\).

    \hypertarget{exercises-calculate-the-fwhm-of-the-central-diffraction-peak-and-the-position-of-the-first-diffraction-side-maximum.-compare-your-results-with-the-simulated-experiment.}{%
\subsection{Exercises: Calculate the FWHM of the central diffraction
peak and the position of the first diffraction side maximum. Compare
your results with the simulated
experiment.}\label{exercises-calculate-the-fwhm-of-the-central-diffraction-peak-and-the-position-of-the-first-diffraction-side-maximum.-compare-your-results-with-the-simulated-experiment.}}

\hypertarget{pinhole-of-size-50-mum.}{%
\subsubsection{\texorpdfstring{Pinhole of size 50
\(\mu\)m.}{Pinhole of size 50 \textbackslash mum.}}\label{pinhole-of-size-50-mum.}}

\hypertarget{pinhole-of-size-10-mum.}{%
\subsubsection{\texorpdfstring{Pinhole of size 10
\(\mu\)m.}{Pinhole of size 10 \textbackslash mum.}}\label{pinhole-of-size-10-mum.}}

\hypertarget{circular-disk-of-50-mum.}{%
\subsubsection{\texorpdfstring{Circular disk of 50
\(\mu\)m.}{Circular disk of 50 \textbackslash mum.}}\label{circular-disk-of-50-mum.}}

\hypertarget{circular-disk-of-10-mum.}{%
\subsubsection{\texorpdfstring{Circular disk of 10
\(\mu\)m.}{Circular disk of 10 \textbackslash mum.}}\label{circular-disk-of-10-mum.}}

\hypertarget{confirm-babinets-principle.-calculate-the-ratio-of-integrated-intensities-between-diffraction-from-an-aperture-and-from-the-circular-disk.}{%
\subsubsection{Confirm Babinet's principle. Calculate the ratio of
integrated intensities between diffraction from an aperture and from the
circular
disk.}\label{confirm-babinets-principle.-calculate-the-ratio-of-integrated-intensities-between-diffraction-from-an-aperture-and-from-the-circular-disk.}}

    \begin{tcolorbox}[breakable, size=fbox, boxrule=1pt, pad at break*=1mm,colback=cellbackground, colframe=cellborder]
\prompt{In}{incolor}{ }{\boxspacing}
\begin{Verbatim}[commandchars=\\\{\}]

\end{Verbatim}
\end{tcolorbox}


    % Add a bibliography block to the postdoc
    
    
    
